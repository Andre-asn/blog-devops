\documentclass[11pt,a4paper]{article}
\usepackage[utf8]{inputenc}
\usepackage[T1]{fontenc}
\usepackage{textcomp}
\usepackage{geometry}
\usepackage{graphicx}
\usepackage{hyperref}
\usepackage{fancyhdr}
\usepackage{lastpage}
\usepackage{datetime}
\usepackage{listings}
\usepackage{xcolor}
\usepackage{titlesec}
\usepackage{tocloft}

% Page geometry
\geometry{margin=1in}
\setlength{\headheight}{14pt}

% Hyperref setup
\hypersetup{
    colorlinks=true,
    linkcolor=blue,
    filecolor=magenta,      
    urlcolor=cyan,
    pdftitle={Blog DevOps Project Documentation},
    pdfauthor={DevOps Team}
}

% Header and footer
\pagestyle{fancy}
\fancyhf{}
\fancyhead[L]{Blog DevOps Project}
\fancyhead[R]{Documentation}
\fancyfoot[C]{\thepage\ of \pageref{LastPage}}
\fancyfoot[R]{Commit: \COMMITHASH}

% Title formatting
\titleformat{\section}{\Large\bfseries}{\thesection}{1em}{}
\titleformat{\subsection}{\large\bfseries}{\thesubsection}{1em}{}

% Code listing style
\lstset{
    basicstyle=\ttfamily\small,
    breaklines=true,
    breakatwhitespace=true,
    frame=single,
    backgroundcolor=\color{gray!10},
    tabsize=2,
    showstringspaces=false
}

% Variables (will be replaced by script)
\newcommand{\COMMITHASH}{af622c9}
\newcommand{\COMMITDATE}{2025-12-10 13:24:08 -0500}
\newcommand{\COMMITAUTHOR}{Andre Santiago-Neyra}
\newcommand{\COMMITMESSAGE}{Merge branch 'main' of https://github.com/Andre-asn/blog-devops}

\begin{document}

% Cover Page
\begin{titlepage}
    \centering
    \vspace*{2cm}
    
    {\Huge\bfseries Blog DevOps Project}\\[1cm]
    {\Large\bfseries Master Documentation}\\[2cm]
    
    \vfill
    
    {\large Commit Hash:}\\[0.5cm]
    {\Huge\ttfamily \COMMITHASH}\\[1cm]
    
    {\large Commit Date: \COMMITDATE}\\[0.5cm]
    {\large Author: \COMMITAUTHOR}\\[0.5cm]
    {\large Message: \COMMITMESSAGE}\\[1cm]
    
    \vfill
    
    {\large Generated: \today\ \currenttime}\\[0.5cm]
    {\large Version: 1.0}\\[1cm]
    
    \vfill
    
    % Optional: Add logo if available
    % \includegraphics[width=0.3\textwidth]{logo.png}
    
\end{titlepage}

% Table of Contents
\newpage
\tableofcontents
\newpage

% Introduction
\section{Introduction}
This document contains the complete documentation for the Blog DevOps Project. It is automatically generated and updated on each commit to the repository.

\subsection{Project Overview}
The Blog DevOps Project is a full-stack blog application demonstrating complete DevOps practices including CI/CD pipelines, automated testing, deployment, monitoring, and infrastructure as Code.

\subsection{Documentation Structure}
This master documentation includes:
\begin{itemize}
    \item Project architecture and design
    \item CI/CD pipeline documentation
    \item Infrastructure setup guides
    \item API documentation
    \item Deployment procedures
    \item Monitoring and observability
    \item Change log
\end{itemize}

% Commit History Section
\section{Commit History}
\subsection{Latest Commit}
\begin{itemize}
    \item \textbf{Hash}: \COMMITHASH
    \item \textbf{Date}: \COMMITDATE
    \item \textbf{Author}: \COMMITAUTHOR
    \item \textbf{Message}: \COMMITMESSAGE
\end{itemize}

\subsection{Changes in This Commit}
\begin{itemize}
\item \texttt{M} \texttt{.github/workflows/ci-cd.yml}
\item \texttt{M} \texttt{Jenkinsfile}
\item \texttt{M} \texttt{README.md}
\item \texttt{D} \texttt{docs/DEMO\_PRESENTATION.md}
\item \texttt{D} \texttt{docs/LOAD\_BALANCER\_SETUP.md}
\item \texttt{D} \texttt{docs/QUICK\_REFERENCE.md}
\item \texttt{D} \texttt{docs/SWOT\_DETAILED.md}
\item \texttt{D} \texttt{docs/TALKING\_POINTS.md}
\item \texttt{R100} \texttt{documentations/actions-doc/README.md	docs/actions-doc/README.md}
\item \texttt{R100} \texttt{documentations/actions-doc/classes\_blog\_app.dot	docs/actions-doc/classes\_blog\_app.dot}
\item \texttt{R100} \texttt{documentations/actions-doc/classes\_blog\_app.png	docs/actions-doc/classes\_blog\_app.png}
\item \texttt{R100} \texttt{documentations/actions-doc/classes\_blog\_app.svg	docs/actions-doc/classes\_blog\_app.svg}
\item \texttt{R100} \texttt{documentations/actions-doc/packages\_blog\_app.dot	docs/actions-doc/packages\_blog\_app.dot}
\item \texttt{R100} \texttt{documentations/actions-doc/packages\_blog\_app.png	docs/actions-doc/packages\_blog\_app.png}
\item \texttt{R100} \texttt{documentations/actions-doc/packages\_blog\_app.svg	docs/actions-doc/packages\_blog\_app.svg}
\item \texttt{R100} \texttt{documentations/jenkins-doc/README.md	docs/jenkins-doc/README.md}
\item \texttt{R100} \texttt{documentations/jenkins-doc/classes\_blog\_app.dot	docs/jenkins-doc/classes\_blog\_app.dot}
\item \texttt{R100} \texttt{documentations/jenkins-doc/classes\_blog\_app.png	docs/jenkins-doc/classes\_blog\_app.png}
\item \texttt{R100} \texttt{documentations/jenkins-doc/classes\_blog\_app.svg	docs/jenkins-doc/classes\_blog\_app.svg}
\item \texttt{R100} \texttt{documentations/jenkins-doc/packages\_blog\_app.dot	docs/jenkins-doc/packages\_blog\_app.dot}
\item \texttt{R100} \texttt{documentations/jenkins-doc/packages\_blog\_app.png	docs/jenkins-doc/packages\_blog\_app.png}
\item \texttt{R100} \texttt{documentations/jenkins-doc/packages\_blog\_app.svg	docs/jenkins-doc/packages\_blog\_app.svg}
\item \texttt{M} \texttt{documentation/master\_documentation.tex}
\item \texttt{M} \texttt{scripts/update\_latex\_documentation.sh}
\end{itemize}


% This section will be populated by the script with commit changes

% Main Documentation Sections
\section{Project Documentation}

This section contains automatically generated documentation including UML diagrams, architecture documentation, and pipeline-generated content.


\subsection{Project README}

\begin{lstlisting}[breaklines=true,breakatwhitespace=true]
\end{lstlisting}

\newpage

\subsection{GitHub Actions   UML Documentation}

\begin{lstlisting}[breaklines=true,breakatwhitespace=true]
\end{lstlisting}

\newpage

\subsubsection{Class Diagram}
The class diagram is available as: \texttt{classes\_blog\_app.png}

\subsubsection{Package Diagram}
The package diagram is available as: \texttt{packages\_blog\_app.png}

\subsection{Jenkins   UML Documentation}

\begin{lstlisting}[breaklines=true,breakatwhitespace=true]
\end{lstlisting}

\newpage

\subsubsection{Class Diagram}
The class diagram is available as: \texttt{classes\_blog\_app.png}

\subsubsection{Package Diagram}
The package diagram is available as: \texttt{packages\_blog\_app.png}


% This section will be populated with documentation from docs/ directory
% Content includes:
% - README.md from project root
% - Generated UML documentation (GitHub Actions and Jenkins)
% - Diagram references and documentation

% Appendices
\appendix
\section{Appendix A: Configuration Files}
% PLACEHOLDER_CONFIG_FILES

\section{Appendix B: Pipeline Logs}
% PLACEHOLDER_PIPELINE_LOGS

\end{document}

